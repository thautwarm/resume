\documentclass{resume}

\usepackage{lastpage}
\usepackage{fancyhdr}
\usepackage{linespacing_fix} % disable extra space before next section

\begin{document}
\pagestyle{fancy}
\fancyhf{}
\renewcommand\headrulewidth{0pt}
\cfoot{\thepage\ of \pageref{LastPage}}

\name{Taine Zhao}

\basicInfo{
  \email{twshere@outlook.com} \textperiodcentered\
  \phone{(+86) 176-0064-7398} \textperiodcentered\
  \github[thautwarm]{https://github.com/thautwarm}
}

\section{\faGraduationCap\ Education}
\datedsubsection{\textbf{Nanjing University of Science and Technology}, Jiangsu, China}{09/15 -- Current}
  Major: Mathematics and Applied Mathematics

\section{\faUsers\ Work Experience}
\datedsubsection{\textbf{Microsoft Research Asia}, Beijing, China}{11/17 -- 02/19}
\role{Intern}{Compiler Design, Distributed System, Machine Learning Framework}
\begin{itemize}
  \item Created \href{https://github.com/Microsoft/GraphEngine/tree/master/src/Modules/GraphEngine.Storage.Composite} {an implementation} of dynamically loading and updating distributed data schema for Microsoft GraphEngine
  \item Created prototypes for MSRA internal projects about future machine learning frameworks, including gluing Python with MySQL in runtime level, and an extended SQL
  \item Created \href{https://github.com/thautwarm/LLAST} {a LLVM IR builder framework} in .NET side, which is introduced into a future/extended version of GraphEngine to achieve JIT support
  \item Finished a part of lowering tasks for the first/basic language in the platform of that future/extended version of GraphEngine
\end{itemize}

\section{\faGithubAlt\ Personal Projects}

\datedsubsection{\textbf{YAPyPy}}{\url{https://github.com/Xython/YAPyPy}}
Yet Another Python Python / Pure Python Compiler
\begin{itemize}
  \item A pure Python compiler built on CPython, which provides full compatibility for multiple versions of CPython 3.x
      , currently support 3.6.x and 3.7.x
  \item Provided some syntax extensions like dictionary destructuring and pattern matching
  \item Capable of customizing parser and bytecode emitter
      , which enables making arbitrary syntactic/semantic extensions with CPython its own
\end{itemize}

\datedsubsection{\textbf{MLStyle.jl}}{\url{https://thautwarm.github.io/MLStyle.jl/latest/}}
A Julia package that provides ML language infrastructures like extensible pattern matching, ADTs/GADTs, etc.
\begin{itemize}
  \item Support a large number of patterns from other languages like Haskell, Elixir, F\#, OCaml, etc.
  \item Provided a group of concise, intuitive and convenient interfaces to customize pattern matching(extensible pattern matching)
  \item Allow users to restrict the accessing of patterns via a customizable predicate in module level 
        , to achieve a clean scoping and avoid conflicts against redefinitions of custom patterns
  \item Provided a homoiconic way to manipulation ASTs via pattern matching 
        , which enables users to extract sub-patterns from given ASTs and rewrite them, without a prerequisite
  about Julia ASTs
\end{itemize}

\datedsubsection{\textbf{FSTan}}{\url{https://github.com/thautwarm/FSTan}}
F\# implementation of Lightweighted Higher Kined Types and type classes
\begin{itemize}
 \item Provided a set of commonly-used abstractions like Functor, Monad, MonadTrans, etc., and some commonly-used data types like State/StateT, Either/EitherT, etc.
 \item Support ad-hoc polymorphisms(via F\#'s STRT), which is an advance comparing to other implementaions
 \item Support separating data type definitions from interfacing/instancing type classes , which is an advance comparing to other implementaions
\end{itemize}

\datedsubsection{\textbf{RSolve}}{\url{https://github.com/thautwarm/RSolve}}
A general purposed solver for logic programming in Haskell
\begin{itemize}
 \item Established an abstraction over unification algorithms, which could be applied to many concrete instances(introduced below)
 \item Provided some concrete instances like HM unification and option question puzzles
 \item No dependency but Haskell standard libraries
\end{itemize}

\datedsubsection{\textbf{LanguageCollections}}{\url{https://github.com/thautwarm/LanguageCollections}}
List of languages invented by me, which has recorded my experience of learning this topic


\section{\faCogs\ Skills}

\subsection{\textbf{Compiler - Front End}}
\begin{itemize}
  \item Capable of creating LL(1) Parser Generator, understand LL(k),  
        experienced in creating lexer generators and Parser Combinators
        . Have several implementations in Python and F\#, one of which supports
        processing context sensitive cases and has some demos about implementing parsers
        for languages that look like Haskell or OCaml.
  \item Have a knowledge about generating LR(1) parsers from BNF grammars or Parser Combinator notations
\end{itemize}

\subsection{\textbf{Compiler - Middle End}}
\begin{itemize}
  \item Experienced in various kinds of (static) program analyses and transformations
        , e.g., 
          implementing custom binary operators with customizable associativities and precedences,
          partial evaluations, forward reference resolutions, lexical/dynamic scoping analysis, syntactic macros, etc.   
  \item Familiar with type inferences based on HM unification
      ,and capable of extending it with
       with row polymorphisms, instance resolutions, GADTs, etc.
\end{itemize}

\subsection{\textbf{Compiler - Back End}}
\begin{itemize}
  \item Familiar with syntaxes, semantics and some intrinsics of LLVM IR
  \item Familiar with CPython bytecode instructions and code objects, etc.
  \item Have some experience about code generation that targets LLVM IR or CPython bytecode
\end{itemize}

\subsection{\textbf{Compiler - Others}}
\begin{itemize}
  \item Experienced in making DSLs
  \item Familiar with low level data layouts
  \item Familiar with implementing high level language constructs
        (Module, Pattern Matching, Switch, Closure, etc.) for both
        compiled languages and interpreted languages.
\end{itemize}

\subsection{\textbf{Functional Programming}}
\begin{itemize}
  \item Familiar with type classes and higher kinded types
        , and have created several implementaions
  \item Understand and can make good use of CPS, Y-Combinators, simple untyped/typed lambda calculus
  \item Understand Monad related stuffs from Monoid to MonadTrans
        , and have a preference of monadic coding style
\end{itemize}

\subsection{\textbf{Machine Learning}}
\begin{itemize}
  \item Used to be familiar with commonly-used DL frameworks like PyTorch and Tensorflow, and capable of picking up again in a few minutes.
  \item Experienced in traditional ML toolchains like NumPy, Scikit-Learn, Pandas, Matplotlib, etc.
  \item Understand forward propogation and back propogation, capable of creating simple neural network frameworks
  \item Understand many traditional ML algorithms like KNN, K-Means, Decision Tree, Random Forest, Stacking, etc.
  \item 2016 CCF/DataFountain Agricultural Product Price Prediction rank 7/500+
  \item Have some knowledge about bioinformatics and NLP
        , familiar with feature extraction methods(PSSM, N-Gram, TF-IDF, etc.)
  \item Capable of taking advantage of ML in daily life
        , such as playing FGO and creating smart CLI(like lightweighted auto-jump)
\end{itemize}

\section{\faInfo\ Miscellaneous}
\begin{itemize}[parsep=0.25ex]
  \item Blog: \url{https://thautwarm.github.io/Site-32/}
  \item PyPI: \url{https://pypi.org/user/thautwarm/}
  \item PyConChina 2018: \url{http://cn.pycon.org/2018/city_beijing.html}, as a lecturer(called \textit{NightyNight}) 
  \item Open source contributions: contributed to some organizations such as \textit{Microsoft, Python} \\
        Fetch the meaningful records from \url{https://thautwarm.github.io/Site-32/Others/contributions.html}
  \item Codewars: \url{https://www.codewars.com/users/NightyNight/}, basically merely prefer to finish the most difficult trainings
\end{itemize}

%% Reference
%\newpage
%\bibliographystyle{IEEETran}
%\bibliography{mycite}
\end{document}

