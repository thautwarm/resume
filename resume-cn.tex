% % !TEX program = xelatex
% This is my resume
% Chinese translation
% originally by ice1000
% adopted by thautwarm

\documentclass{resume}

\usepackage{lastpage}
\usepackage{fancyhdr}
\usepackage{linespacing_fix} % disable extra space before next section
\usepackage[fallback]{xeCJK}

%% \setmainfont[]{SimSun}
%% \setCJKfallbackfamilyfont{rm}{HAN NOM B}
% \setCJKmainfont[BoldFont=Sarasa Gothic SC,ItalicFont=KaiTi_GB2312]{Source Han Serif SC}
%% \renewcommand{\thepage}{\Chinese{page}}

\begin{document}
\pagestyle{fancy}
\fancyhf{}
\renewcommand\headrulewidth{0pt}
\cfoot{\thepage\ of \pageref{LastPage}}

\name{赵王宏楦}

\basicInfo{
  \email{twshere@outlook.com} \textperiodcentered\
  \phone{(+86) 176-0064-7398} \textperiodcentered\
  \github[thautwarm]{https://github.com/thautwarm}  \textperiodcentered\
  \linkedin[赵王宏楦]{https://www.linkedin.com/in/thautwarm/}  \textperiodcentered
}

\section{\faGraduationCap\ 教育经历}
\datedsubsection{\textbf{南京理工大学}}{2015.9 -- 2019.6}
  专业:数学与应用数学

\datedsubsection{\textbf{筑波大学}}{2019.12 -- 至今}
  专业:计算机科学

\section{\faUsers\ 工作经历}
\datedsubsection{\textbf{微软亚洲研究院}, 北京, 中国}{2017.11 -- 2019.2}
\role{实习}{编译器设计, 分布式系统, 机器学习框架}
\begin{itemize}
    \item 分布式的数据schema动态加载更新机制\href{https://github.com/Microsoft/GraphEngine/tree/master/src/Modules/GraphEngine.Storage.Composite}{实现}
    \item MSRA内部原型实现。关于未来机器学习框架; 在运行时层面进行Python和MySQL胶合的扩展SQL
    \item 纯.NET端的\href{https://github.com/thautwarm/LLAST} {LLVM IR的Builder框架}, 被未来版本的Graph Engine引入以支持JIT特性
    \item 负责未来版本的Graph Engine上第一语言的一部分底层代码生成工作
\end{itemize}


\section{\faHeartO\ 常规}
\begin{itemize}[parsep=0.25ex]
\item {\textbf{院青年志愿者协会 - 副会长, 2016}}  时年在南京理工大学周边社区组织、参与了长期、大量的社会服务活动, 孝陵卫街道办颁布"社区优秀大学生志愿者"荣誉

\item {\textbf{江苏数学专业数学竞赛二等奖, 2016}} 时年在学习程序设计之余,偶然参与该竞赛并获奖

\item {\textbf{南理工理学院"院长奖章", 2017}} 该奖章为该校院级学生最高荣誉, 2017年首届颁发
\item {\textbf{专业认可, 2019}} \\
经过长年累月的学习和训练,于2019年3月获得本人崇拜多年的程序语言教授Oleg Kiselyov的荐举,到日本筑波大学进行深造学习。\\
同时也受到筑波大学教授兼本科计算机主任Yukiyoshi Kameyama的称赞和升学方面的帮助。

\end{itemize}

\section{\faGithubAlt\ 个人专业项目}


\datedsubsection{\textbf{Restrain-JIT}}{\url{https://github.com/thautwarm/restrain-jit}}
在CPython发行版下, 一个真的能用的Python JIT实现. 提供了运行示例, 但仍处于开发阶段.

\datedsubsection{\textbf{moshmosh}}{\url{https://github.com/thautwarm/moshmosh}}
Python语法扩展库, 提供模式匹配

\datedsubsection{\textbf{YAPyPy}}{\url{https://github.com/Xython/YAPyPy}}
另一个Python实现的Python,支持在Python端优化、扩展语法和语义

\datedsubsection{\textbf{MLStyle.jl}}{\url{https://thautwarm.github.io/MLStyle.jl/latest/}}
引入ML(Meta Language)的模式匹配, 代数数据类型等高级特性的Julia语言程序库

\datedsubsection{\textbf{FSTan}}{\url{https://github.com/thautwarm/FSTan}}

F\#语言实现的轻量级高阶类型(Light-Weighted Higher Kinded Types)以及类型类(Type class)

\datedsubsection{\textbf{RSolve}}{\url{https://github.com/thautwarm/RSolve}, \url{https://github.com/thautwarm/rsolve.py}}
Haskell/Python中用于逻辑编程的通用求解器

\datedsubsection{\textbf{CanonicalTraits.jl}}{\url{https://github.com/thautwarm/CanonicalTraits.jl}}
Julia中, 真正的Typeclass实现

\datedsubsection{\textbf{GeneralizedGenerated.jl}}{\url{https://github.com/thautwarm/GeneralizedGenerated.jl}}
利用编译技术放宽了Julia staged编程技术的一些实现限制, 极大扩展Julia的staged编程能力.

特别的, 对绝大多数的Julia抽象语法树, 提供了相比静态定义没有开销的运行时eval. 这一技术在其他大多数语言中是反直觉且不可迄及的.

% \begin{itemize}
%  \item Established an abstraction over unification algorithms, which could be applied to many concrete instances(introduced below)
%  \item Provided some concrete instances like HM unification and option question puzzles
%  \item No dependency but Haskell standard libraries
% \end{itemize}

\datedsubsection{\textbf{LanguageCollections}}{\url{https://github.com/thautwarm/LanguageCollections}}
一些个人创造的程序语言集合, 以记录相关方面的学习经历

% Reference Test
%\datedsubsection{\textbf{Paper Title\cite{zaharia2012resilient}}}{May. 2015}
%An xxx optimized for xxx\cite{verma2015large}
%\begin{itemize}
%  \item main contribution
%\end{itemize}

%% \section{\faHeartO\ 成就}
%% \datedline{}{Aug. 2017}


\section{\faCogs\ 专业技能}

\subsection{\textbf{编译器 - 前端}}
\begin{itemize}
    \item 非常熟悉Parser组合子相关技术及一些还未大量应用的学界提出的扩展
    \item 非常熟练地实现LL(k), 熟悉大量学界前沿的相关扩展, 熟悉LR(1), 了解mildly context-sensitive parsing
    \item 个人实现的技术上集大成(LL(k)及其前沿扩展)的parser generator: \url{https://github.com/thautwarm/RBNF.hs}
\end{itemize}


\subsection{\textbf{编译器 - 中端}}
\begin{itemize}
  \item 熟练掌握大量静态程序分析以及变换的方法
  \item 熟悉类型推导, 熟练掌握HM类型推导方法, 常作扩展.
  \item 了解如何将高级语言的构造(模式匹配, 闭包, 模块等)编译到低级语言
  \item 熟悉DSL制作流程且充满热情
\end{itemize}

\subsection{\textbf{编译器 - 后端}}
\begin{itemize}
  \item 熟悉LLVM IR指令集,包括相关语法、语义和内部固有函数(intrinsics)
  \item 熟悉MIPS指令集
  \item 熟悉CPython的字节码指令集,包括字节码对象
  \item 有一些编译到LLVM IR, CPython字节码或MIPS汇编的实际项目经历
  \item 了解底层内存模型, 能够精准剖析C/C++代码的运行时行为
\end{itemize}

\subsection{\textbf{函数式编程}}
\begin{itemize}
  \item 熟悉Type class, 高阶类型以及相关实现
  \item 理解并能很好的应用CPS变换、一些不动点组合子, lambda calculus等
  \item 理解Monad相关概念,从Monoid(幺半群)到MonadTrans
  \item 了解一些更高级的概念例如程序证明、依赖类型等,为依赖类型语言Idris语言实现了Python和Julia语言的后端: \url{https://github.com/thautwarm/idris-cam}
\end{itemize}


\subsection{\textbf{机器学习}}
\begin{itemize}
  \item 熟悉多个Python机器学习工作栈,尤其熟悉并爱好数据特征工程
  \item 2016 CCF大数据农产品价格预测排名7/500+
  \item 有生物信息学和自然语言处理的知识背景,有相关的"搬砖"和商业项目经历
  \item 能够灵活地在日常生活中应用机器学习,智能地处理包括电子游戏、命令行交互在内的事项
\end{itemize}

\section{\faInfo\ 其他}
\begin{itemize}[parsep=0.25ex]
  \item 博客: \url{https://thautwarm.github.io/Site-32/}, 暂无中文
  \item PyPI: \url{https://pypi.org/user/thautwarm/}
  \item Julia中国Meetup 2019: \url{https://github.com/JuliaCN/MeetUpMaterials/tree/master/Beijing2019/thautwarm}
  \item 中国Python开发者大会, PyConChina 2018: \url{http://cn.pycon.org/2018/city_beijing.html}, 作为讲师参与(名称为 \textit{NightyNight})
  \item 中国Python开发者大会, PyConChina 2019: \url{http://cn.pycon.org/2019/index.html}, 作为讲师参与(名称为 \textit{thautwarm})
  \item 开源贡献: 对一些组织如\textit{Microsoft, Python, Julia}进行了贡献 \\
        其中一些重要的记录见于\url{https://thautwarm.github.io/Site-32/Others/contributions.html}
  \item 获取此简历的最新版本: \url{https://raw.githubusercontent.com/thautwarm/resume/master/resume-cn.pdf}
\end{itemize}


%% Reference
%\newpage
%\bibliographystyle{IEEETran}
%\bibliography{mycite}
\end{document}


